\documentclass[a4paper, 12pt]{article}
% polish language settings
\usepackage[polish]{babel}
\usepackage[T1]{fontenc} 
\usepackage[utf8]{inputenc}
\usepackage{graphicx}
\usepackage{hyperref}
\usepackage[usenames]{color}
\usepackage{listings}
\graphicspath{ {./images/} }

\definecolor{mGreen}{rgb}{0,0.6,0}
\definecolor{mGray}{rgb}{0.5,0.5,0.5}
\definecolor{mPurple}{rgb}{0.58,0,0.82}
\definecolor{backgroundColour}{rgb}{0.95,0.95,0.92}

\lstdefinestyle{CStyle}{
    backgroundcolor=\color{backgroundColour},   
    commentstyle=\color{mGreen},
    keywordstyle=\color{magenta},
    numberstyle=\tiny\color{mGray},
    stringstyle=\color{mPurple},
    basicstyle=\footnotesize,
    breakatwhitespace=false,         
    breaklines=true,                 
    captionpos=b,                    
    keepspaces=true,                 
    numbers=left,                    
    numbersep=5pt,                  
    showspaces=false,                
    showstringspaces=false,
    showtabs=false,                  
    tabsize=2,
    language=C
}

% project info
% document settings
\title {
	\begin{center}
	\includegraphics[width=0.2\textwidth]{godlo}
	
	\vspace{1cm}
	
	\large {
		\textbf {
			POLITECHNIKA ŚLĄSKA
		}

		\vspace{1cm}

		Projekt Inżynierski
	}

	\vspace{1cm}
	
	\LARGE {\textbf {Projekt i wykonanie układu pomiarowego do analizy jakości powietrza dla UAV}}
	
	\vspace{5cm}

	\end{center}
}

% authors
\author{
	Autor: Michał Matysiak\\
	
	\vspace{1cm}
	
}

% date
\date {
	\begin{center}
	\textit{Gliwice, Styczeń 2021}
	\end{center}
}

\begin{document}
% title-page

\maketitle
\newpage

% table of contents page
\pagenumbering{roman}
\tableofcontents
\pagenumbering{arabic}
\newpage

% Wstep
\section{Wstęp}
\hspace{0.6cm}Wstęp
\newpage
% Cel
\section{Cel}
\hspace{0.6cm}Cel

% Założenia
\section{Założenia}
\hspace{0.6cm}Założenia

\newpage
% Harmonogram
%\section{Harmonogram}

%\subsection{Tydzień 1}
%\begin{itemize}
%\item Prosta symulacja działania układu.
%\item Zakup potrzebnych komponentów.
%\item Zaprojektowanie przekładni, która umożliwi nam poruszanie roletą.
%\end{itemize}

%\subsection{Tydzień 2}
%Tudzien 2

%\newpage

% Schemat blokowy
\section{Schemat blokowy}
\hspace{0.6cm}Schemat blokowy
\begin{center}
\includegraphics[width=0.7\textwidth]{godlo}\\
\textit{Rys. [1] Diagram ideowy}
\end{center}



\newpage
% schemat elektryczny
\section{Schemat elektryczny}


% porównanie
\section{Alternatywy} 
\hspace{0.6cm}Alternatywne rozwiazania 



\subsection{Zalety}
\noindent Zalety projektu:
\begin{itemize}
\item Cena

\end{itemize}

\noindent Zalety projektów komercyjnych:
\begin{itemize}
\item Wysoka jakość wykonania
\item Serwis
\item Gwarancja
\end{itemize}

\subsection{Wady}
\noindent Wady projektu:
\begin{itemize}
\item Nieestetyczny wydruk 3D
\item Potrzeba kalibracji
\item Potrzeba minimalnej wiedzy do uruchomienia
\end{itemize}

\noindent Wady projektów komercyjnych:
\begin{itemize}
\item Bardzo wysoka cena
\item Nie pasują do każdego okna
\item Sterowanie z pilota
\item Długi czas oczekiwania na dostawę i montaż 
\end{itemize}

\newpage
% Podzespoły elektroniczne
\section{Podzespoły elektroniczne}


\newpage
%Konstrukcja Mechaniczna
\section{Projekt obudowy}
\hspace{0.6cm}


\newpage
%Rzeczywiste etapy pracy
\section{Przebieg prac}
\hspace{0.6cm}Postęp prac

\newpage
%Oprogramowanie
\section{Oprogramowanie}
\subsection{ESP}

\hspace{0.6cm}
\begin{lstlisting}[style=CStyle]
KOD
\end{lstlisting}

%Testy
\section{Testy}
\hspace{0.6cm}testy
\newpage  

%Problematyka projektu
\section{Problematyka projektu}
\hspace{0.6cm}
Problematyka


%Podsumowanie
\section{Podsumowanie}
\subsection{Osiągnięte cele}
\hspace{0.6cm}Podsumowanie
%Możliwości rozwojowe
\subsection{Możliwości rozwojowe}
\hspace{0.6cm}MR

%Alternatywne drogi projektu
\subsection{Alternatywne drogi}
\hspace{0.6cm}Alternatywne drogi
\newpage
%Literatura
\section{Literatura}
\end{document}